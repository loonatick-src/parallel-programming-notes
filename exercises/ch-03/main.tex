\documentclass[a4paper]{article}

\usepackage[utf8]{inputenc}
\usepackage[T1]{fontenc}
\usepackage{textcomp}
\usepackage[english]{babel}
\usepackage{amsmath, amssymb}
\usepackage{caption}
\usepackage{minted}


% figure support
\usepackage{import}
\usepackage{xifthen}
\pdfminorversion=7
\usepackage{pdfpages}
\usepackage{transparent}
\newcommand{\incfig}[1]{%
	\def\svgwidth{\columnwidth}
	\import{./figures/}{#1.pdf_tex}
}
\newcommand{\hcut}{\hbar}

\newenvironment{code}{\captionsetup{type=listing}}{}

\DeclareMathSymbol{\C}{\mathalpha}{AMSb}{"43}
\DeclareMathSymbol{\R}{\mathalpha}{AMSb}{"52}

\pdfsuppresswarningpagegroup=1

\title{Exercises: Distributed Memory Programming with MPI}
\date{}

\begin{document}
\maketitle
\section*{1}
\subsection*{Greeting}
Changing \texttt{strlen(greeting) + 1} to \texttt{strlen(greeting)}
would exlude the terminating null character from the greeting string.
But, as it turns out, when passing the non-null-terminated string
to \texttt{MPI\_Send}, MPI checks for the terminating null character
and adds one if one is not already there. So, everything works as 
intended.

If \texttt{MAX\_STRING} is passed instead, then the message might include
bytes beyond the terminating null character. But, string functions like
\texttt{printf} et cetera should work just fine because that includes
the terminating null character.

\texttt{mpicc} also uses the same options as \texttt{gcc}, so
conditional preprocessing can be done to produce three variants of
the code by using appropriate \texttt{gcc} options.

\begin{code}
\inputminted[samepage=false, breaklines, linenos, firstline=1]{c}{codes/ex3_1.c}
\label{normal}
\caption{Greetings program with preprocessor directives}
\end{code}

Use \texttt{mpicc -D NONULL -o nonull ex3\_1.c} to get the code in
listing \ref{nonull}.

\begin{code}
\inputminted[samepage=false, breaklines, breakanywhere, linenos=true, firstline=8709]{c}{codes/ex3_1_nonull.c}
\label{nonull}
\caption{Without the terminating null character}
\end{code}



Similarly, using \texttt{mpicc -D MAX\_STRING -o maxstring ex3\_1.c}
gives.

\begin{code}
\inputminted[breaklines, linenos, firstline=8709]{c}{codes/ex3_1_fullbuffer.c}
\end{code}

\subsection*{Trapezoid Rule}


\end{document}
